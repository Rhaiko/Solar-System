%----------------------------------------------------------------------------------------
%	SUMMARY.
%----------------------------------------------------------------------------------------

\section*{Summary}

There are two main theories about the origin of water on Earth: one theory suggests that the inner Solar System was too hot during the creation of planets so no water could aggregate during the formation of the Earth. Therefore, the water must have been brought to Earth after it was formed, presumably by comets and asteroids. While asteroids only contain a maximum relative water amount of approximately 10\%, comets on the other hand mainly consist of water ice (relative water amount usually exceeds 80 or 90\%). During the bombardment of the Earth these objects then have brought particles such as H$_2$O from the depths of the Solar System to Earth. 
The other theory does not exclude the possibility of water particles being brought to Earth with asteroids or comets, however, it claims that the amount of water brought to Earth by extraterrestrial sources by far not enough to fill the oceans, cover the polar caps with ice as well as all the rest of the water that exists on the surface or is stored subterraneously. Therefore, the Earth must have accredited wet.\\

To prove these theories the D/H ratio of water on Earth is compared with the D/H ratio of asteroids and comets. Due to the fact that these objects originate from different parts of the Solar System, their D/H ratios differ, as proven by many measurements on different extraterrestrial bodies as well as the Earth. The results show that the the D/H ratio of an object increases with the distance from the sun of the point of origin. Therefore, the D/H ratio on Earth is smaller than the D/H ratio usually found on comets. However, there are discrepancies in this model as the D/H ratio of Venus is a lot higher than the D/H ratio of the Earth despite the fact that it is closer to the Sun. In addition to this, there are comets such as 103P/Hartley 2 that are believed to originate from the Kuiper belt and have D/H ratios very similar to Earth. These large discrepancies are not yet covered by an updated model. Also, for further research it is essential to measure the D/H ratios of other comets and asteroids to be able to refine these models.\\

As most of the comets have a D/H ratio far higher than that of Earth, the favored theory would be that the Earth accredited wet and only a small fraction of the Earth's water originates from comets and asteroids. Nonetheless, the measurement results from comet 103P/Hartley 2 show very similar results to those on Earth so it is also likely that the Earth accredited dry and gained all its water from comets and asteroids. Before additional research on more asteroids and comets is conducted it is not possible to favor one theory over another.



