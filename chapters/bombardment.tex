%----------------------------------------------------------------------------------------
%	BOMBARDMENT OF EARTH.
%----------------------------------------------------------------------------------------

\section{\label{chap:bombardment}The bombardment of Earth}
There are differing trains of thought on the timing, how and where the Earth received its water. The possibilities where water could come from are either through internal reserves acquired during the accretion process or from extraplanetary sources after Earth’s development. These extraplanetary sources of water would have come from asteroids, comets and protoplanet collisions. The hypothesis that concerns these objects delivering water to a developed and dry Earth is referred to as the late bombardment hypothesis. The hypothesis that Earth’s water comes from extraplanetary origins is not flawless and has controversy. There is another hypothesis where the water was accumulated during the Earth’s accretion development. This hypothesis is referred to as the “Accretion Model”. These are the two base theories about the origin of Earth’s water. There are other theories which are derived from the base theory with changes. These hypotheses have evolved over time to account for previously unknown scientific data.\\

The late bombardment hypothesis has been addressed by different names such as “Late Veneer”, “Late Bombardment Hypothesis”, “Late Accretion”, and “Late Heavy Bombardment”: for consistency within this report, this hypothesis will be referred as “Late Bombardment Hypothesis” or LBH. The late bombardment hypothesis has variations that cover the various objects that could deliver water to the Earth as well as the influences of different planetary orbits. The various objects included are comets, asteroids and planetesimals. This hypothesis was first noted by Hopp\cite{BOMB1} but not named as such in an article from Kimura et al. concerning the discrepancy in gold distribution \cite{BOMB2}. However, another article from Maruyama \cite{BOMB3} report that another article written by E. Anders \cite{BOMB4} first came up with the theory. The start and progression of this theory concerns the siderophile elements and their distribution within the Earth. The elements include tungsten, ruthenium, gold, osmium, iridium, and more elements \cite{BOMB5} \cite{BOMB6} \cite{BOMB7}. These reports on the different elements shows the gathering of data for this hypothesis. The late bombardment hypothesis has contending views on the origin of Earth’s water. The most popular to least popular opinions are asteroids, comets and then planetesimals. There are reports and articles \cite{BOMB8} \cite{BOMB9} \cite{BOMB10} that highlight the differences between the sources as well as likelihood and the effects coming from the different sources. The most commonly discussed of these sources are the carbonaceous chondrite asteroids. The asteroids are the viewed as the most likely source of the Earth’s Water in the late bombardment hypothesis.\\

The hypothesis that states that Earth had contained water while formation was occurring is known as the “Accretion Model”. This theory also has other names such as “Wet Earth” and “Cool Earth”, for consistency and clarity. This hypothesis will further be addressed as “Early Accretion Model”, to distinguish itself from the previous hypothesis. This model has been proposed in opposition to the late bombardment hypothesis. This hypothesis was started because of the isotopic differences between Earth and extraplanetary objects. The hypothesis is based on the accretion of volatile elements and water in grains of dust and being able to retain them during the development of the planet. The developments in the hypothesis comes from showing that water can be absorbed into different materials at high temperatures of the accretion disc. There has been a study showing that olivine can be hydrated in high temperatures \cite{BOMB11} \cite{BOMB12}. A new article \cite{BOMB13} discusses evidence that there was significant water before the collision that formed the moon, which is before the time period of the LBH. This hypothesis has not been as deeply researched as the late bombardment hypothesis has been.\\

The main two hypotheses are the late bombardment hypothesis and the early accretion model. The late bombardment hypothesis says that Earth was formed dry and received water and volatiles from extraplanetary sources. This model dismisses all claims from the early accretion model concerning the possibility of the Earth forming wet. The early accretion model does not preclude the late bombardment hypothesis, however does severely limit the amount of water accumulated from extraplanetary sources. There is agreement on “our knowledge of detail of terrestrial planet formation and hydration is currently insufficient” \cite{BOMB14}. \\

\subsection{Main 2 - Late Bombardment Hypothesis}
\subsubsection{Section 1 - Comets}
Par 1 - Intro\\
Par 2 - Calculations and probability for needed quantity to reach earth\\
Par 3 - Any other side notes or\\ effects on Earth’s composition\\
Par 4 - Controversy, short comings and failures\\
Par 5 - Conclusion\\

\subsubsection{Section 2 - Asteroids}
Par 1 - Intro -\\
Par 2 - Calculations and probability\\ for needed quantity to reach earth\\
Par 3 - Any other side notes or effects on Earth’s composition\\
Par 4 - Controversy, short comings and failures\\
Par 5 - Conclusion\\

\subsubsection{Section 3 - Planetesimals}
Par 1 - Varieties, starting locations, general compositions and ratios, hydration levels\\
Par 2 - Calculations and probability for needed quantity to reach earth\\
Par 3 - Any other side notes or effects on Earth’s composition\\
Par 4 - Controversy, short comings and failures\\
Par 5 - Conclusion\\

\subsubsection{Section 4 - ABEL Model}
Par 1 - Comparison to the other Late\\ Bombardment Hypothesis methods\\
Par 2 - Implications of the model\\
Par 3 - Controversy and shortcomings\\

\subsection{Main 3 - Early Accretion Model}
Section 1 – Accretion Theory\\
Par 1 - Intro\\
Par 2 - Reiterate possibility of keeping water\\
Par 3 - Evidence for some accretion\\
Par 4 - Controversy, short comings and failures\\
Par 5 - Conclusion\\

\subsection{Main 4 - Conclusion}
Par 1 - Reiterate D/H Ratios\\
Par 2 - Which theories can account for the D/H Ratio\\
Par 3 - Try to reconcile the two theories\\
Par 4 - Conclusion\\