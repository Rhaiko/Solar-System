\documentclass{article}
\usepackage[utf8]{inputenc}
\usepackage{cite}

\include{ref.bib}

\title{Origins of Water on Planet Earth}
\author{adahoe-8 }
\date{October 2018}

\begin{document}

\maketitle

There are differing trains of thought on the timing, how and where the Earth received its water. The two possibilities where water could come from are either through internal reserves acquired during the accretion process or from extraplanetary sources after Earth’s development. These extraplanetary sources of water would have come from asteroids, comets and protoplanet collisions. The hypothesis that concerns these objects delivering water to a developed and dry Earth is referred to as the late heavy bombardment. The hypothesis that Earth’s water comes from extraplanetary origins is not flawless and has controversy. There is another hypothesis where the water was accumulated during the Earth’s accretion development. This hypothesis is referred to as the “Accretion Model”. These are the two base theories about the origin of Earth’s water. These hypotheses have evolved over time to account for previously unknown scientific data.

\section{Early Accretion Model}
The hypothesis that states that Earth had contained water while formation was occurring is known as the “Accretion Model”. This theory also has other names such as “Wet Earth” and “Cool Earth”, for consistency and clarity. This hypothesis will further be addressed as “Early Accretion Model”, to distinguish itself from the previous hypothesis. This model has been proposed in opposition to the Late Heavy Bombardment. This hypothesis was started because of the isotopic differences between Earth and extraplanetary objects. The hypothesis is based on the accretion of volatile elements and water in grains of dust and being able to retain them during the development of the planet. The developments in the hypothesis comes from showing that water can be absorbed into different materials at high temperatures of the accretion disc. There has been a study showing that olivine can be hydrated in high temperatures \cite{BOMB11} \cite{BOMB12}. A new article \cite{BOMB13} discusses evidence that there was significant water before the collision that formed the moon, which is before the time period of the LHB. This hypothesis has been evaluated in the previous section and evaluated the claims of the ability of Earth to retain early water. This hypothesis has not been as deeply researched as the Late Heavy Bombardment has been.

\section{Late Heavy Bombardment}
The Late Heavy Bombardment has been addressed by different names such as “Late Veneer”, “Late Bombardment Hypothesis”, “Late Accretion”, and “Late Heavy Bombardment”: for consistency within this report, this hypothesis will be referred as “Late Heavy Bombardment” or LHB. The late heavy bombardment has variations that cover the various objects that could deliver water to the Earth as well as the influences of different planetary orbits. The various objects included are comets, asteroids and planetesimals. The time frame of this bombardment happened between 4.4 and 3.5 Ga \cite{LHB_Timeframe}. This hypothesis was first noted by Hopp\cite{BOMB1} but not named as such in an article from Kimura et al. concerning the discrepancy in gold distribution \cite{BOMB2}. However, another article from Maruyama \cite{BOMB3} report that another article written by E. Anders \cite{BOMB4} first came up with the theory. The start and progression of this theory concerns the siderophile elements and their distribution within the Earth. The elements include tungsten, ruthenium, gold, osmium, iridium, and more elements \cite{BOMB5} \cite{BOMB6} \cite{BOMB7}. These reports on the different elements shows the gathering of data for this hypothesis. The late heavy bombardment has competing views on the origin of Earth’s water. The most popular to least popular opinions are asteroids, comets and then planetesimals. There are reports and articles \cite{BOMB8} \cite{BOMB9} \cite{BOMB10} that highlight the differences between the sources as well as likelihood and the effects coming from the different sources. The most commonly discussed of these sources are the carbonaceous chondrite asteroids. The asteroids are the viewed as the most likely source of the Earth’s Water in the late heavy bombardment.

\subsection{Comets}

\subsubsection{Introduction}
Comets have showered Earth in dust and ice in their wake and the occasionally impact on Earth. 
There are two classes of comets: short period and long period comets. 
The distinction is whether their orbital period is less or greater than 200 years. 
The long period comets are from the Kuiper Belt and the Oort Cloud while the short period comets are from the Jupiter and Saturn Belt.
With the discovery of comets comprising significantly of water, comets were the first extraplanetary objects to be considered as the source of Earth's water. The impact of comets could deliver significant amount of water and would effect Earth's climate and atmosphere.

\subsubsection{Probability and Numbers}

Comets are abundant in the solar system. Estimates of the number of comets in the solar system are in the magnitudes of \(10^{11}\sim 10^{12}\) objects \cite{Comet_Num}.The current amount of comets is approximately one-fifth of the original amount of comets \cite{Comet_AvgSz}. The amount of comet material could amount to almost \(2\%\) \cite{Comet_AvgSz} the solar system mass. However scientists have only been able to observe under \(6000\) \cite{observed_objects} comets and explored and studied only a handful of comets.

The mass of comet 67P is \(9.98 \times 10^{12}\)kg \cite{67P_Mass} with water ice accounting for \(75\%\) \cite{67P_Water} of the comet which corresponds to \(7.48 \times 10^{12} kg\).
Comet Halley is \(2.2 \times 10^{14}kg\) \cite{Cevolani1987} and contains \(40\% \) water ice \cite{Comet_Yeomans} by weight. The comet Halley should have \(8.8 \times 10^{13} kg\) of water ice.
Assuming average spherical size and density of short period comets are 4.4 kilometers \cite{comet_AvgSz} and \(0.6 \: \sfrac{g}{cc}\) \cite{Britt2006SmallBD} respectively meaning an average mass of \(2.14 \times 10^{14}kg\). Assuming similar hydration levels as come Halley with 40\% water ice by weight, an average comet would hold about \(8.56 \times 10^{13}kg\) kg of ice.

Under ideal conditions of perfect transfer of all material from the comet to the Earth and no loss of previously accumulated water, it would take almost 180 million comets to provide Earth its \(1.54\times 10^{22} kg\) of water. A total of \(3.85\times 10^{22} kg\) of comets would be needed for this set of ideal conditions.

Assuming non-ideal conditions, such as, an non perfect transfer on water and non-negligible of previous water  due to the impact. The amount of water accumulated will not reach the desired value of \(1.54\times 10^{22} kg\), this is due to the loss of previous water being equal to the amount of water transferred to Earth as the assumed average comet is not large enough to transfer more water than the Earth would have lost to space.

\subsubsection{Controversy and Disagreements}
The hypothesis of comets being the source of water on Earth is not without contention. The main points that cause scientist to disagree with this assertion are the deuterium to hydrogen ratios and the time frame of the bombardment from Jupiter's comets.
The comet deuterium hydrogen ratios are much higher than Earth's, especially with comet 67P being 3 times the value of Earth's. 
The second point that causes disagreements which includes Jupiter's development into the equation of scattering comets and asteroids towards Earth. Comets were most likely scattered within a million years after Jupiter was fully developed, and at that point of time in the solar system, the Earth was a much smaller target \cite{morbidelli2000source}. 
These arguments against tend to discount comets being a main source of water for Earth.


\subsubsection{Conclusion}
Comets are bountiful objects of ice and dust that people throughout the ages can admire or fear. They did deliver ice and other volatile elements to Earth through collisions.
There is enough cometary mass in the solar system to provide Earth the required amount of water. 
However due to the issue with the size of the planet at the time of bombardment and the high deuterium to hydrogen ratio, scientists have stated that this is improbable source.
The high deuterium to hydrogen ratio found in comets, some scientists speculate that at most 10\% of the Earth's water comes from comets \cite{morbidelli2000source}.
Comets have contributed to Earth's water but in a minor role rather than the primary contributor.

\subsection{Asteroids}
\subsubsection{Introduction}
Asteroids are becoming the popular origin of water on Earth.
Asteroids are different from comets, both in origin location and composition.
The asteroid belt can be separated by the the Kirkward gaps at 2.5 and 2.82 AU\cite{AsteroidBelt}, into the inner, middle and the outer asteroid belt.
These inner belt has differing properties compared to the middle and outer belts with the most notable in this report being hydration values.

\subsection{Calculations and Associated Probability}
The inner asteroid belt is much dryer with an hydration of \(<0.1\%\) by weight while the  asteroid belt beyond 2.5 AU contains more water with up to \(10\%\) by weight \cite{morbidelli2000source}. The sizes of asteroids range from meters to hundreds of kilometers large with the asteroid belt being composed of asteroids smaller and larger than 1 kilometer in size \cite{asteroid_AvgSz}. The average density of an asteroid depends on its classification, the densities varies from \(1.38 \: \sfrac{g}{cc}\) for carbonaceous, \(2.71 \: \sfrac{g}{cc}\) for silicate, and \(5.32 \: \sfrac{g}{cc}\) for metallic asteroids\cite{Asteroid_Density}. 

With an estimate of 1 kilometer sized asteroid with an averaged density of \(3.1 \: \sfrac{g}{cc}\)and assuming the hydration to be \(0.1\%\) for inner belt asteroids and \(10\%\) for middle and outer belt asteroids.
In Ideal transfer of perfect mass capture and negligible previously retained water escaping due to the collision, the amount of inner belt asteroids to collide with Earth would be just under 1 trillion asteroids to give Earth its \(1.54\times 10^{22} kg\) of water. This amount of asteroids is problematic as it is more than twice the mass of planet Earth at \(1.5\times 10^{25} kg\).
It would take 10 billion middle and outer belt asteroids with a combined mass of \(1.5\times 10^{23} kg\) or \(2.5\%\) of Earth's mass.
The current asteroid belt has \(3.58 \times 10^{21}kg\) \cite{AsteroidBeltMass}.
It is hypothesized that the current asteroid belt is a remnant compared to the primordial asteroid belt at 1\% its former size \cite{PrimAstBelt}. If using the former size, there is sufficient material if all the asteroids were well hydrated.

\subsubsection{Controversy and Disagreements}
There is not a consensus on if asteroids is the origin of water on Earth. One of the issues that cause contention is Jupiter's influence on the asteroid belt. Jupiter's influence on the scattering of the primordial asteroid belt would have been too early. Morbidelli concluded that the large majority of asteroids would collide with the Earth before the Earth was 60\% formed \cite{morbidelli2000source}. 
There is also a contention on the primordial asteroid belt size, with a new hypothesis stating that it was empty and was filled instead of losing nearly all of its objects \cite{Emptybelt}. This would prevent asteroids being the source of water as there would have not been any to deliver water to the Earth.
The primordial belt if two magnitudes larger than today's value would have a similar variation of hydration and would not deliver the needed amount of water on Earth.

\subsubsection{Conclusion}
Asteroids are considered the leading explanation for water on Earth. 
The deuterium to hydrogen ratio of asteroids are much closer to Earth's values than comet's. 
However this scenario still leaves much to desire as the primordial asteroid belt might not have significant enough amount of water to hydrate the Earth.

\subsection{ABEL Model}
\subsubsection{Introduction}
The new hypothesis in Late Heavy Bombardment is ``Advent of Bio-ELements" (ABEL) model\cite{ABEL_1}. 
ABEL attempts to reconcile evidence of water within the Earth with LHB. ABEL model comprises of a two step process: Earth was formed dry with no atmosphere or oceans then gained an atmosphere and oceans. 

\subsubsection{Differences between ABEL and other LHB}
ABEL model considers multiple geological characteristics compared to the previously described Late Heavy Bombardment scenarios. 
The late bombardment that occurred in the middle of Hadean era is renamed to the ABEL bombardment.
ABEL model considers the origin of plate tectonics and the presence of water within the Earth which is not considered within the previous models. 
The areas that ABEL differs from that of the cometary, asteroidal or planetesimal scenarios are having no atmosphere, a rigid lithosphere, and a completely reductive planet \cite{ABEL_1}.

ABEL model states that the Earth originally formed without an atmosphere. The atmosphere was developed during the ABEL bombardment with the presence of volatiles and water. The bombardment provided the oceanic and atmospheric elements to the dry Earth \cite{ABEL_2}.

A large concentration of ABEL is the plate tectonics. The model proposes that there was a large solid lithosphere \cite{ABEL_2}. The ABEL bombardment with massive asteroids broke and subducted the primordial continents into the mantle as well as  introducing volatiles which is needed for plate tectonics \cite{ABEL_3}. 

\subsection{Effects of Late Heavy Bombardment}
The impacts from extraplanetary object have significant effects on the planet Earth.
The consistent impacts consistent with LHB would produce large lava oceans, up to 10\% of Earth's surface  and would have increased the temperature significantly and possibly evaporate all the oceans \cite{LavaOcean}.
The impacts would have caused massive amounts of rock dust and water vapor to enter the high atmosphere. The sub-micrometer dust cloud can severely block incoming light from the sun and cool the planet, and the water vapor may be photolyzed and be lost to space \cite{ImpactDust}. 
These are very harsh environments for life to form from.


\section{Conclusion}

The main two hypotheses are the Late Heavy Bombardment and the early accretion model. The Late Heavy Bombardment says that Earth was formed dry and received water and volatiles from extraplanetary sources. This model dismisses all claims from the early accretion model concerning the possibility of the Earth forming wet. The early accretion model does not preclude the Late Heavy Bombardment, however does severely limit the amount of water accumulated from extraplanetary sources. There is agreement on “our knowledge of detail of terrestrial planet formation and hydration is currently insufficient” \cite{BOMB14}. 

\section{Bibliography}
\bibliography{ref}{}
\bibliographystyle{plain}
\end{document}




